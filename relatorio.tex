 %formatacao do documento e tipo do documento
\documentclass[12pt, a4paper, twoside]{article}

%pacotes de extensoes
\usepackage[portuges]{babel} %pkg da lingua portugues
\usepackage[latin1,utf8]{inputenc} %pkg da lingua portugues
\usepackage{titlepic} %inserir imagens na capa
\usepackage{graphicx}
\usepackage[table,xcdraw]{xcolor} %usar fomatacao nas tabelas
\usepackage{geometry}\geometry{a4paper=true,portrait=true,left=3cm,right=3cm,top=2.5cm,bottom=3.5cm}
\usepackage{multicol}
\usepackage{hyphenat}
\usepackage{booktabs}
\usepackage{array}
\usepackage{indentfirst}

\makeindex

%inicio do documento
\begin{document}

%inserir capa / titulo
\title{
Departamento de Informática\\Licenciatura em Tecnologias de Informação\\
\vskip 1 cm
\includegraphics[height=5 cm]{logo_fcul.png}
\vskip 1 cm
Planeamento e Gestão de Projecto\\
\vskip 1 cm
Relatório Projecto individual}
\author{André Nunes\\43304}
\date{}

\maketitle

\newpage
%indice
\tableofcontents

\pagenumbering{roman}
\setcounter{page}{1}

\newpage
\pagenumbering{arabic}
\setcounter{page}{1}
\section{Introdução}
\vskip 1 cm

Este projeto tem como objetivo proporcionar ao aluno uma primeira experiência de planeamento e gestão de um projeto de software. Nesse sentido, a ênfase da execução do projeto deve ser colocada nas tarefas de planeamento e gestão, e não nas tarefas de desenvolvimento (ou seja, não se espera que o produto de software entregue no final do projeto apresente elevados níveis de eficácia e eficiência).\\

\textbf{Produto}\\

Pretende-se que o aluno desenvolva uma aplicação Web para registo de objetos emprestados pelo utilizador. No mínimo, a aplicação deve realizar as seguintes funcionalidades:
\begin{itemize}
\item registar um empréstimo de um objeto a indicar, numa data a indicar, a uma pessoa a indicar;
\item registar a devolução de um objeto emprestado, numa data a indicar;
\item ver uma lista de objetos emprestados atualmente.
\end{itemize}

O aluno poderá acrescentar mais funcionalidades se assim o entender. A aplicação Web não deve fazer uso de uma base de dados externa.

\newpage
\section{Planeamento das tarefas}
\vskip 1 cm

Este projeto foi dividido em cinco fases de implementação:

\\
\begin{enumerate}
	\item Desenvolver Base
  	\begin{itemize}
		\item Servidor Flask
  		\item Página inicial (html + css)
  	\end{itemize}
  	\item Funcionalidade Registar Empréstimo (Reg Emp)
  	\begin{itemize}
  		\item Html + css (a)
  		\item Javascript (b)
  		\item Ligação ao servidor (c)
	\end{itemize}
	\item Funcionalidade Registar Devolução (Reg Dev)
  	\begin{itemize}
  		\item Html + css (a)
  		\item Javascript (b)
  		\item Ligação ao servidor (c)
	\end{itemize}
	\item Funcionalidade Listar Empréstimos (Listar Emp)
  	\begin{itemize}
  		\item Html + css (a)
  		\item Javascript (b)
  		\item Ligação ao servidor (c)
	\end{itemize}
	\item Desenvolver Base
  	\begin{itemize}
  		\item Testes
  		\item Documentação
	\end{itemize}
\end{enumerate}

\newpage
Para cada tarefa foi planeado um tempo de execução, apresentado na tabela seguinte:

\newcolumntype{P}[1]{>{\centering\arraybackslash}p{#1}}
\newcolumntype{L}[1]{>{\arraybackslash}p{#1}}
\begin{table}[h]
\centering
\begin{tabular}{L{4cm}|P{2.5cm}|P{2.5cm}}

Tarefa 			& Tempo Planeado (horas) & Tempo Usado (horas) \\ \hline
Servidor Flask  & 2              & 0.96           \\
Página incial   & 2              & 1.33           \\
Reg Emp (a)     & 2              & 0.76           \\
Reg Emp (b)     & 2              & 0.86           \\
Reg Emp (c)     & 2              & 1.16           \\
Reg Dev (a)     & 2              & 0.53           \\
Reg Dev (b)     & 2              & 0.46           \\
Reg Dev (c)     & 2              & 1.53           \\
Listar Emp (a)  & 2              & 0.95           \\
Listar Emp (b)  & 2              & 0.50           \\
Listar Emp (c)  & 2              & 0.50           \\
Testes          & 2              & 1.63           \\
Documentação    & 1              & 0.93           \\ \hline
Total           & 25             & 12.1

\end{tabular}
\\~\\
\caption{Tabela de tempo de cada tarefa}
\label{table:tabela_tempo}
\end{table}

\clearpage
\newpage
\section{Desenvolvimento}
\vskip 0.5 cm

Depois de uma análise rápida às tecnologias necessárias para desenvolver o projeto, escolhi desenvolver o servidor em python 3 utilizando a framework Flask devido à sua simplicidade, rápida elaboração e experiência prévia com a framework. A parte do design da página web foi feita apenas com uma tabela e dois botões para interagir com o utilizador, algo simples de elaborar e funcional. Quando é executada alguma das ações de empréstimo ou retorno de um objeto a aparece uma confirmação ao utilizador e a tabela é atualizada automaticamente.
De modo a tornar a aplicação web persistente, tenho um ficheiro csv que é usado como base de dados para guardar o registo dos empréstimos. Esta base de dados é atualizada sempre que há um empréstimo ou uma devolução. 

\subsection{Instalação}

Para a utilização deste projeto é necessário ter o python3 e o flask instalado. O passo seguinte é navegar até a pasta do projeto e fazer "python3 app.py".

\subsection{Tempo de desenvolvimento}

Um dos objetivos deste projeto individual era termos uma noção do planeamento que é preciso fazer antes da fase de desenvolvimento, algo fulcral num projeto de média/grande dimensão. No início do projeto é definido um plano e uma ordenação das tarefas a elaborar, com tempos bem definidos para cada tarefa, facto que deve ser levado em conta no mundo profissional. Portanto, antes de começar a 
"martelar" código foi necessário elaborar a Tabela \ref{table:tabela_tempo} com uma estimativa do tempo necessário para cada tarefa, com base em projetos anteriormente realizados. Esse planeamento foi feito usando o software Project 2016 da Microsoft. \par Depois de o desenvolvimento estar finalizado constatei que o tempo que foi usado para a realização do projeto foi inferior ao tempo planeado. Após de finalizar a funcionalidade "registar um empréstimo" foi muito mais rápido implementar as outras funcionalidades, visto que usavam código parecido. \par O tempo previsto para a elaboração do projeto foi de 25 horas, o tempo real utilizado foi de 11 horas e 9 minutos (12.1 horas). No final do projeto posso concluir que poderia haver um ajuste no tempo para cada tarefa, dado que o projeto é relativamente simples.

%fim do documento
\end{document}