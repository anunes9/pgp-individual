%Modelo Descentralizado- Especificar como a equipa vai funcionar.
%Conclusão do COCOMO

%formatação do documento e tipo do documento
\documentclass[12pt, a4paper, twoside]{report} %inicio do doc

%pacotes de extensões
\usepackage[portuges]{babel} %pkg da lingua portugues
\usepackage[latin1,utf8]{inputenc} %pkg da lingua portugues
\usepackage{verbatim} %pkg para escrever sem formataçao
\usepackage{color} %usar cores nas letras
\usepackage{graphicx} %usar imagens no doc
\usepackage[table,xcdraw]{xcolor}
\usepackage{makeidx}
\usepackage{anysize} % para formatar o tamanho do documento
\usepackage{multirow}
\usepackage{mathtools}
\usepackage{wrapfig}
\usepackage{footnote}

\marginsize{3.17cm}{3.17cm}{2.54cm}{2.54cm}

%fazer índice
\makeindex

\begin{document}

\title{%
	\textbf{Planeamento e Gestão de Projecto}\\ 
	\large 3º Relatório
}

\author{%
Francisco Pires, nº 44314 \\
Alexandre Machado, nº 43551 \\
}

\date{\today}
\maketitle
%mostrar indice
\tableofcontents

%--------------------------------------------------------------------------------------------- Feito
\chapter{Introdução}

Este projecto tem como objectivo o desenvolvimento e a implementação de um Sistema de Informação (SI) para uma plataforma de partilha de música e venda de bilhetes para concertos.
\\

\noindent Após consulta de serviços \textit{web} de \textit{streaming} de música (\textit{spotify}), divulgação directa de música pelos artistas (\textit{bandcamp} e \textit{soundcloud}) e venda de bilhetes para eventos (\textit{ticketline} e \textit{ticketmaster}), decidimos propor um projecto que congregue numa plataforma única, as funcionalidades disponibilizadas por estes serviços, e tentando propor novos requisitos funcionais para melhor enquadrar o SI.
\\

\noindent Este SI é baseado em tecnologias \textit{web}, nas quais vamos procurar dar ênfase no uso de dispositivos \textit{mobile}. A razão  para o reforço das tecnologias \textit{mobile} prende-se, por um lado, com o uso cada vez mais generalizado destes dispositivos, e por outro lado, porque o âmbito do projecto se adequa à implementação destes requisitos funcionais nos referidos dispositivos.

%--------------------------------------------------------------------------------------------- Feito

\chapter{Análise de requisitos}

\section{Requisitos funcionais e não funcionais}

\subsection{Requisitos funcionais}

De acordo com os objectivos definidos para este projecto, selecionamos as seguintes funcionalidades que o utilizador terá disponíveis neste SI.

\begin{enumerate}

\item \textit{Upload} de conteúdos vídeo e áudio gratuito ou com opção de pagamento
	\begin{itemize}
	\item Singles, albuns, videoclips, curtas e \textit{playlists}
	\end{itemize}
	
\item Criação e manutenção de eventos
    \begin{itemize}
    \item Divulgação e venda de entradas para eventos ao vivo
    \item Local, data, e participantes
    \end{itemize}
    
\item Registo e manutenção de um perfil
	\begin{itemize}
	\item Criação  e alteração da \textit{password}, username, heterónimos e informação pessoal
	\end{itemize}

\item Registo e manutenção de perfil de bandas/artistas
    \begin{itemize}
    \item Ferramenta de publicação de conteúdos
    \item Listagem de artistas participantes na banda
    \item Eventos em que vão participar e editoras e/ou promotores associados
    \end{itemize}
    
\item Registo e manutenção de perfil de editoras e promotores
	\begin{itemize}
	\item Editoras listam todas as bandas e edições publicadas
    \item Promotores listam eventos com uma ou várias bandas
	\end{itemize}
	
\clearpage
	
\item Partilha de eventos \textit{On-The-Go}
	\begin{itemize}
	\item Eventos que não foram previamente anunciados, que estão a acontecer no momento da publicação
	\item O utilizador assinala a localização do evento e escreve uma breve descrição sobre ele
    \item Possível associação de uma banda a um evento \textit{On-The-Go}
	\end{itemize}
	
\item Modelo de \textit{follow}
	\begin{itemize}
	\item Permite receber notificações sobre determinadas bandas/eventos
	\end{itemize}
	
\item Partilha de conteúdo com outros utilizadores (\textit{share/listen})
    \begin{itemize}
    \item Definição da visibilidade do conteúdo.
    \end{itemize}
    
\item Recomendações profissionais (artistas a recomendar outros artistas)

\item Página principal - \textit{mural}
    \begin{itemize}
    \item Aparece o conteúdo partilhado por perfis que seguimos
    \end{itemize}
    
\item Sistema de recomendações automático (\textit{if you like this then you may like that})

\item \textit{Sound System}
    \begin{itemize}
    \item Sincronizam a música para fazer um sistema de som, com vários dispositivos na mesma rede.
    \end{itemize}
    
\item \textit{Show Me}
	\begin{itemize}
	\item Mostrar que conteúdo está a ser ouvidos pelos utilizadores de dispositivos mobile num determinado raio de proximidade
    \item Definição de visibilidade por terceiros
	\end{itemize}
	
\item Para as opções monetárias, usar \textit{satoshi}
    \begin{itemize}
    \item um \textit{satoshi} é uma ínfima parte de uma \textit{bitcoin} (1 \textit{satoshi}	= 0.00000001 \textit{bitcoin})
    \item Usada com o objectivo de criar um registo de transações sem valor comercial
    \item Apenas para efeitos de desenvolvimento. Substituído por bitcoin ou outros meios de pagamento em fase de produção

    \end{itemize}
    
\item Partilha de conteudos e gestão de \textit{logins} atraves da integração com o \textit{Facebook} e \textit{Twitter}

\end{enumerate}


%--------------------------------------------------------------------------------------------- Feito

\subsection{Requisitos não funcionais}

Para execução das funcionalidades neste SI, será necessário assegurar os requisitos não funcionais que listamos de seguida.

\begin{enumerate}
\item Música por \textit{cache}
    \begin{itemize}
    \item A música que se encontra na mesma rede local (em cache ou em memória), é partilhada usando \textit{p2p} para não sobrecarregar o servidor.
    \end{itemize}
\item Confidencialidade dos dados
	\begin{itemize}
	\item Nas transferências de dinheiro relativas às compra de bilhetes
	\item No conteúdo partilhado pelos artistas em modo restrito (para uma lista de utilizadores previamente definidos)
	\end{itemize}
\item Segurança dos dados e dos acessos
	\begin{itemize}
	\item O acesso aos dados de um utilizador especifico vai ser permitido mediante escolha da visibilidade desse conteúdo (pública, restrita e privada).
	\end{itemize}
\item Garantia de disponibilidade
	\begin{itemize}
	\item Garantir que a plataforma está sempre acessível online (24/7).
	\end{itemize}
\item Escalável e modular
	\begin{itemize}
	\item Capacidade de poder aumentar a capacidade de servir um maior número de clientes, iniciando paralelamente mais instâncias do servidor de aplicação e de \textit{web}
	\end{itemize}
\item Tempo de resposta
	\begin{itemize}
	\item O servidor terá a resposta pronta num tempo inferior a 500ms
	\end{itemize}
\item Persistência, sincronização dos dados e disponibilidade
	\begin{itemize}
	\item A informação guardada nos vários servidores vai ser distribuída por várias instâncias. Estas instâncias vão ser sincronizados automaticamente para não haver falhas de persistência e de sincronização, e para alem disso, vai permitir uma rápida recuperação para o SI poder tolerar falhas de hardware
	\end{itemize}
\item Notificações e alertas de acontecimentos do utilizador
\begin{itemize}
\item Estas serão enviadas para os utilizadores via email ou telemóvel
\end{itemize}
\item \textit {Responsive Web Design}
\begin{itemize}
\item Garantir que o SI pode ser acedido por dispositivos \textit{mobile} e que nestas plataformas os requisitos funcionais definidos anteriormente funcionem correctamente
\end{itemize}
\end{enumerate}

\section{Modelo de casos de uso}
\subsection{Casos de Uso textuais}

\subsubsection{Compra de Bilhetes}
\\
\\
Ator Principal: Utilizador
\\
Interesses: O utilizador pretende comprar um bilhete para um evento
\\
\\
Pré-condições: O utilizador está registado na plataforma.
\\
Pós-condições: O utilizador adquire um bilhete para o evento.
\\
\\
Cenário principal de sucesso:
\begin{enumerate}

\item O utilizador dirige-se a página do evento.
\item O sistema devolve a informação sobre o evento(preço, localização, descrição)
\item O utilizador indica que quer comprar um bilhete para o evento.
\item O sistema redireciona o utilizador para uma página segura com informação sobre o evento que deseja comprar bilhete e quanto bilhetes deseja comprar.
\item O utilizador indica o número de bilhetes que quer comprar.
\item O sistema devolve o total atualizado conforme o nº indicado e com o stock disponível.
\item O utilizador indica que quer prosseguir com a compra.
\item O sistema pergunta que tipo de pagamento o utilizador vai utilizar
\begin{enumerate}
\item O utilizador indica que vai utilizar Cartão de Credito.
\item O sistema devolve os campos necessários para efetuar a compra com o método selecionado(NºCartão de Credito,CVV,Data de Validade,NIF)
\item O utilizador indica que vai utilizar Satoshi
\item O sistema devolve um código QR e um endereço com a carteira e o montante a pagar a ser utilizado pelo cliente para transferir os satoshi’s
\item O utilizador completa o pagemento utilizado o método que preferir
\item O sistema informa que o pagamento foi efetuado com sucesso e envia para a conta de email registada no utilizador com os bilhetes adquiridos.
\end{enumerate}
\end{enumerate}




Cenários alternativos: 
	3b. O sistema indica ao utilizador que o evento se encontra esgotado.
	5b. O sistema indica ao utilizador que o nº de bilhetes pedido não se encontra disponível.
	9b. O sistema informa que a tentativa de pagamento falhou, e indica para este tentar outra vez





\chapter{Planeamento}

\section{Recursos}

\textbf{Recursos Humanos}
\\

Os recursos humanos para o projecto incluem seis alunos de Tecnologias de Informação (LTI), sendo que os quatro alunos não presentes neste relatório são ficticios. 
No final da cadeira de Planeamento e Gestão do Projecto (PGP), os dois grupos irão juntar-se e trabalhar em conjunto nas cadeiras de Projecto Tecnologias de Informação (PTI) e Projecto Tecnologias de Redes (PTR). A duração total do projecto será de sete meses, sendo três meses e meio dedicados ao planeamento (PGP).
\\
\\
\textbf{Disponibilidade}
\\

A disponibilidade dos alunos é conforme apresentada na seguinte tabela:

\begin{table}[h]
\centering
\begin{tabular}{|l|c c|}
\hline
\multirow{2}{*}{} & \multicolumn{2}{c|}{Disponibilidade} \\ \cline{2-3} 
                  		& 1ºSemestre        & 2ºSemestre       \\ \hline
Colega Edgar \footnotemark      & 20\%              & 28,6\%             \\ \hline
Colega Esperança \footnotemark  & 20\%              & 33,3\%           \\ \hline
Colega Emídio \footnotemark     & 20\%              & 40\%           \\ \hline
Francisco Pires                 & 20\%              & 28,6\%           \\ \hline
Alexandre Machado               & 20\%              & 28,6\%           \\ \hline
Colega Eva\footnotemark         & 20\%              & 33,6\%                \\ \hline
\end{tabular}
\caption{Tabela de Disponibilidade}
\label{disponibilidade}
\end{table}

\footnotetext{Os alunos em questão são fictícios, visto que não vão efectuar
a cadeira de PTI/PTR. A disponibilidade foi simulada por valores reais.}

\clearpage

%---------------------------------------------------------------------------------------------

\noindent{\textbf{Organização da equipa}}
\\
\\
A organização dos membros da equipa vai ser feita em três grupos. Os grupos constituídos são um grupo para PTR, outro para PTI e um último grupo para os “elementos móveis”.Estes alunos vão contribuir em conjunto para o trabalho de ambas as cadeiras. A decisão de organizar o projecto distribuído em três grupos surgiu para dar a resposta ao facto de dois membros terem competências equivalentes em PTI, PTR e disponibilidade acrescida para as necessidades de cada departamento do projeto.

\begin{itemize}
\item Grupo PTR
\begin{itemize}
	\item Francisco Pires
	\item Colega Edgar
\end{itemize}
\item Grupo PTI
\begin{itemize}
	\item Colega Eva
	\item Colega Emidio
\end{itemize}
\item \textit{Elementos Moveis}
\begin{itemize}
	\item Alexandre Machado
	\item Colega Esperança
\end{itemize}
\item Gestor de Projecto
\begin{itemize}
	\item Alexandre Machado
\end{itemize}
\end{itemize}

\noindent{\textbf{Tabela de Competências}}
%COMPLETAR AS COLUNAS DE DESIGN E RH
\begin{table}[h]
\centering
\begin{tabular}{|l|c c c c c c c|}
\hline
                  & PHP & Java & HTML & CSS & Python & Interface & Gestão   \\ \hline
Colega Emidio     & 3   & 4    & 3    & 2   & 3      & 2         & 1        \\ \hline
Colega Eva        & 3   & 3    & 3    & 3   & 3      & 3         & 2        \\ \hline
Colega Edgar      & 3   & 3    & 3    & 3   & 3      & 3         & 1        \\ \hline
Francisco Pires   & 2   & 4    & 4    & 3   & 4      & 2         & 2        \\ \hline
Alexandre Machado & 2   & 4    & 3    & 3   & 4      & 3         & 3        \\ \hline
Colega Esperança  & 2   & 4    & 3    & 3   & 2      & 3         & 1        \\ \hline
\end{tabular}
\caption{Tabela de Competências}
\label{competencias}
\end{table}

\clearpage

%---------------------------------------------------------------------------------------------

\section{Estimação}

Para a realização da tabela relativa aos dados históricos, foram escolhidas as cadeiras em que a matéria dos projectos se encaixa no âmbito do projecto.

\begin{table}[h]
\centering
\begin{tabular}{|l|c c c c c c|}
\hline
                  & AD        & ASW       & ITW     & ADS     & SO     & PTI/PTR \\ \hline
Alexandre Machado & 1002/160h & 2576/160h & 756/72h & 454/18h & 560/42h  & 4910/380h\\ \hline
Francisco Pires   & 1002/160h & NA        & 687/5h  & NA      & 775/50h  & 4910/380h \\ \hline
\end{tabular}
\caption{ Dados Históricos (\textit{Lines of Code} e horas).}
\label{my-label}
\end{table}

\subsection{Esforço disponível}

\begin{itemize}

\item 1º semestre (duração: 3,5 meses)
\begin{equation}
20+20+20 = 60 \ (0,6 \ pessoas)
\end{equation}
\begin{equation}
E = 0,6 \ . \ 3,5 = 2,1 \ PM
\end{equation}
\item 2º semestre (duração: 3,5 meses)
\begin{equation}
28,6+28,6+40+33,3+28,6+33,6 = 192,7 \ (1,927 \ pessoas)\\
\end{equation}
\begin{equation}
\ E = 1,88 \ . \ 3,5 = 6,58 \ PM
\end{equation}
\item Total
\begin{equation}
\ E = 2,1 \ + \ 6,58 = 8,68 \ PM
\end{equation}
\end{itemize}

\subsection{Linhas de código}

As linhas de código previstas para o projecto foram são conforme apresentadas na seguinte tabela. Para o calculo final das LOC, foi usada a seguinte equação:
\\
\\

\begin{equation} S = \ \bigg({Sotim + 4 . Sprov + Spess \over 6}\bigg)
\end{equation}

\clearpage

\begin{savenotes}
\begin{table}[h]
\centering
\begin{tabular}{|l|c c c c|}
\hline
                       					& Optimista  & Provável & Pessimista & \textbf{Final}\\ \hline
Criar Modelo da Base de Dados(NoSQL)  	& 200       & 400      & 600        & 400   		 \\ \hline
Configurar Nginx \textit{HTTP Server} 	& 5         & 20       & 50         & 22    		 \\ \hline
Implementar CDN com API \footnote{CDN - \textit{Content Delivery Network} para melhor distribuição de conteudos}
										& 100       & 350       & 500       & 333    		 \\ \hline
Sistema de P2P \footnote{\textit{Peer-to-Peer} para o requisito de música em \textit{cache}}
                                        & 800       & 1100      & 1600      & 1134   	     \\ \hline
Sistema Distribuído	                    & 100       & 270       & 400       & 264   		 \\ \hline
\textit{Views}            				& 1500      & 2000      & 3000      & 2084  		 \\ \hline
Controlador                 			& 500       & 850       & 1100      & 834   		 \\ \hline
Modelo                      			& 50        & 150       & 250       & 150   		 \\ \hline
Sistema de Pagamento                    & 100       & 250       & 400       & 250   		 \\ \hline
Configurações de Rede                   & 20        & 40        & 60        & 40     		 \\ \hline
Construção dos Contentores(\textit{Dockerfile})
                                        & 40        & 65        & 100       & 67       		 \\ \hline
\textit{Continuous Intergration}        & 100       & 150       & 250       & 159            \\ \hline
Servidor Aplicacional                   & 5         & 10        & 15        & 10   		     \\ \hline
\textbf{Total}		   					& 3520      & 5655      & 8325      & \textbf{5747}  \\ \hline
\end{tabular}
\caption{Linhas de Código}
\label{codigo}
\end{table}%
\end{savenotes}


%---------------------------------------------------------------------------------------------

\subsection{Modelos Empíricos}

Calculo do esforço orgânico:
\\

\begin{equation}
E = a \ . \ KLOC ^ b
\end{equation}

\begin{equation} E = 2,4 \ \bigg({N.Linhas \over 1000}\bigg)^{1.05}
\end{equation}

\begin{equation}
{E = 2,4 \ \bigg({5747 \over 1000}\bigg) ^ {1.05}}
= 15,05 \ P.M
\end{equation}

Calculo da Duração:

\begin{equation}
D = c \ . \ E^d
\end{equation}

\begin{equation}
D = 2,5 \ (15,05)^{0,38}= 7\ M
\end{equation}
\\

\clearpage
\section{Processo de Desenvolvimento de Software}

Como processo de desenvolvimento do nosso projeto decidimos usar o Processo Unificado.
Esta decisão foi baseada numa reflexão da nossa parte, em que, pensámos na forma como trabalhamos e, visto que este projeto não é de forma alguma \textit{full-time}, tivemos de ter isso em conta. O Processo Unificado permite-nos avançar iterativamente e ao mesmo tempo voltar a trás sem que hajam muitos problemas, havendo assim um balanço entre o avançar no projeto e ajustar problemas anteriores, o que achamos que seria perfeito no nosso caso.
\\
%\includegraphics[scale=0.6]{image1.png}

\begin{figure}[h!]
  \centering
    \includegraphics[width=0.6\textwidth]{image1.png}
   \caption{Exemplo de um Processo Unificado}
\end{figure}


\textbf{O Processo Unificado divide-se em três fases:}
\\

\begin{enumerate}
\item \textit{Inception} – justifica-se a execução do projeto, ou seja, tenta-se adquirir um conhecimento do que irá ser preciso para concluir o projeto e quando concluído, os resultados deste.

\item \textit{Elaboration} – conclui-se de certa forma a fase de \textit{inception}, visitando com mais detalhe todos os fatores de risco, \textit{reward} e recursos que este irá trazer. Convém ser o mais completo e detalhado possível visto que na fase seguinte vai proceder-se à construção do projecto.

\item \textit{Construction \& Transition} – começa-se a construção do que irá ser uma versão operacional do projeto. O foco principal nesta fase é a construção de features discutidas anteriormente. É de valor notar que em projetos de maior dimensão esta fase poderá ter varias iterações. Também nesta fase será feita a transição do ambiente de desenvolvimento para um ambiente de produção, pondo o projecto disponível ao cliente final. De seguida compara-se o estado do projecto nesta fase à fase de \textit{Inception} e se tudo estiver bem, faz-se uma \textit{release}. 


\end{enumerate}
%---------------------------------------------------------------------------------------------

\textbf{Vantagens do Processo Unificado:}
\begin{itemize}
\item O cliente não precisa de esperar muito tempo para entrar em contacto com um resultado prático.
\item Quando terminado o desenvolvimento do projeto é muito dificil encontrar erros dada a facilidade de os corrigir anteriormente.
\item Os riscos de grau mais elevado são trabalhados em primeiro lugar, dando assim alguma confiança no desenvolvimento do projeto
\end{itemize}

\textbf{Desvantagens do Processo Unificado:}
\begin{itemize}
\item Poderá haver desorganização em períodos mais avançados no projeto.
\item Aumento de gastos em implementações de varias versões do projetos.
\end{itemize}

%---------------------------------------------------------------------------------------------

%\section{Planeamento do Projecto}

\section{Gestão de Riscos}

Nesta avaliação dos riscos para o nosso projeto, identificámos que existem três grandes áreas: a de Relações Humanas, a de Tecnologia e a de Desenvolvimento do projeto.\\\\ Na categoria de RH identificámos que os principais problemas têm a haver com a relação entre membros do grupo e o comportamento de cada um. \\\\Consideramos estes riscos bastante importantes visto que uma má dinâmica de grupo pode arruinar o potencial de um projeto.\\\\ Na categoria de Tecnologia identificámos que bugs e a segurança são os principais riscos a ter em conta e, iremos dar ênfase à segurança no projeto, visto que uma das partes mais criticas de qualquer sistema.\\\\ Por ultimo, na categoria de desenvolvimento do projeto, identificámos que, sem surpresa, o maior problema são os atrasos que poderão acontecer, podendo estragar planos e horários planeados para a completação do projeto.\\\\ Em geral achamos que os nossos riscos irão ser de natureza comum a todos os grupos, são riscos que a maioria dos projetos, quer a nível académico ou profissional, encontram, não significando que os podemos levar menos a serio, sendo esta a causa de projetos falhados em varias áreas.

\clearpage
\subsubsection{Riscos Relativos aos Recursos Humanos (RM)}
\begin{enumerate}
\item Má comunicação
\item Falta de empenho
\item Falta de conhecimento
\item Baixa de um membro do grupo
\item Falta de “química” entre membros do grupo
\item Atraso na entrega de trabalho de um membro do grupo
\item Desistência de um membro do grupo
\end{enumerate}
\subsubsection{Riscos Relativos às Tecnologias (T)}
\begin{enumerate}
\setcounter{enumi}{8}
\item Má implementação (bugs) de uma funcionalidades
\item Updates que pioram o funcionamento de funcionalidades
\item Falta de segurança do projeto
\end{enumerate}
\subsubsection{Riscos Relativos ao Cumprimento do Planeamento, e à Entrega (E)}
\begin{enumerate}
\setcounter{enumi}{10}
\item Atrasos na entrega do projeto
\item Falta de funcionalidades na entrega do projeto final
\item Requisitos incompletos
\end{enumerate}

\clearpage

\subsubsection{Tabela de Riscos Ordenada:}

\noindent Na “Probabilidade”, a escala escolhida foi de 0% a 100%.
No impacto, a escala escolhida foi de 1 a 5, onde 1 significa
“Baixo impacto” e 5 “Alto impacto” no Produto Final.
\\
\noindent Organizamos agora a tabela por impacto x probabilidade.

\begin{table}[h]
\centering
\begin{tabular}{|l|l l l l|}
\hline
Riscos              & Tipo             & Probabilidade & Impacto & P*I \\ \hline
10                  & Tecnologia       & 80\%   & 3 & 2,4   \\ \hline
3                   & Recursos Humanos & 70\%   & 3 & 2,1   \\ \hline
12 					& Desenvolvimento  & 90\%   & 2 & 1,8   \\ \hline
5             		& Recursos Humanos & 60\%   & 3 & 1,8   \\ \hline
8        			& Tecnologia       & 80\%   & 2 & 1,6   \\ \hline
6                   & Recursos Humanos & 70\%   & 2 & 1,4   \\ \hline
2                   & Recursos Humanos & 50\%   & 2 & 1,0   \\ \hline
4                   & Recursos Humanos & 50\%   & 2 & 1,0   \\ \hline
11              	& Desenvolvimento  & 90\%   & 1 & 0,9   \\ \hline
13                  & Desenvolvimento  & 70\%   & 1 & 0,7   \\ \hline
1                   & Recursos Humanos & 30\%   & 2 & 0,6   \\ \hline
7                   & Recursos Humanos & 20\%   & 3 & 0,6   \\ \hline
9 					& Tecnologia       & 30\%   & 2 & 0,6   \\ \hline
\end{tabular}
\caption{Tabela de Riscos}
\label{riscos}
\end{table}

\noindent Com a Regra de Pareto (80/20) aplicada à gestão de riscos, temos:

\begin{equation}
13 \ riscos \ no \ total \ * \ 0,2 = 2,6
\end{equation}

\noindent Colocaríamos a linha vermelha abaixo dos 3 primeiros riscos, mesmo sabendo que o 4º risco tem a mesma Probabilidade*Impacto, visto que a falta de "química" entre membros do grupo é um problema inexestente para um grupo que só vai fazer o planeamento de um projecto.

\subsubsection{RMMM – Risk, Mitigation, Monitoring \& Management}

Os Project Managers terão mais atenção aos riscos acima da linha vermelha.
\\
\\
\noindent \underline{\textbf{Risco}}: Falta de segurança do projeto
\\
\\
\textbf{Mitigação}:
\begin{itemize}
\item Adquirir bons conhecimentos previamente.
\item Na fase de desenvolvimento, pensar primeiro na segurança, visto que iremos
tratar com informações sensíveis.
\item Usar os conhecimentos da cadeira de Segurança Informática para consolidar a construção do sistema.
\item Testar intensamente, com recurso a DDoS's simulados e tentativas de acessos indevidos.
\item Definir uma configuração de firewall usando \textit{iptables}.
\end{itemize}
\\
\\
\textbf{Monitorização}:
\begin{itemize}
\item Efetuar testes numa fase final de cada iteração.
\item Vigiar o sistema para se poder detectar possíveis ataques ao sistema, recorrendo a Sistemas de Detecção de Intrusões como o \textit{snort}
\item Usar ferramentas de monitorização de ataques maliciosos como o \textit{nmap}.
\end{itemize}
\\
\\
\textbf{Gestão}:
\begin{itemize}
\item Tentar perceber a natureza e o impacto da falha, para uma melhor gestão.
\item Num caso simples, efectual um \textit{patch} de segunraça, conhecendo a vunerabilidade.
\item Num caso mas complicado, retirar a aplicação do ambiente de produção para desenvolvimento até resolver o problema.
\end{itemize}
\\
\\
\noindent \underline{\textbf{Risco}}: Falta de conhecimento
\\
\\
\textbf{Mitigação}:
\begin{itemize}
\item Começar a estudar as tecnologias o mais cedo possivel.
\item Recorrer aos docentes, de apoio ao projeto ou e se necessário outros, que nos poderão ajudar a adquirir os conhecimentos requeridos para a realização do projecto
\end{itemize}
\\
\\
\textbf{Monitorização}:
\begin{itemize}
\item Rever e testar o código a cada funcionalidade.
\item Vigiar o estado do futuro update, e se necessário, fazer alterações a algumas frameworks utilizadas
\item Fazer testes regulares.
\end{itemize}
\\
\\
\textbf{Gestão}:
\begin{itemize}
\item Redobrar o esforço no inicio de cada iteração.
\item Não implementar algumas funcionalidades de menor importância, se as tecnologias utilizadas forem demasiado complexas.
\end{itemize}
\\
\\
\noindent \underline{\textbf{Risco}}: Falta de funcionalidades na entrega do projeto final
\\
\\
\textbf{Mitigação}:
\begin{itemize}
\item Fazer metas intercalares.
\item Planear para terminar o projeto antes do prazo de entrega.
\end{itemize}
\\
\\
\textbf{Monitorização}:
\begin{itemize}
\item Fazer reuniões semanais para verificar se o projeto cumpre as metas intercalares
\end{itemize}
\\
\\
\textbf{Gestão}:
\begin{itemize}
\item Aumentar o esforço da equipa.
\item Alocar a equipa para desenvolver as funcionalidades em profundidade e não em largura.
\end{itemize}
\\
\\
\clearpage

\chapter{Recursos de hardware e software}

Alguns destes pontos vão ser mais detalhados na próxima entrega (capítulo de arquitectura)

\subsubsection{Recursos de software:}

\begin{itemize}
\item Sistema Operativo: CentOS 7
\item Servidor Web: Nginx (redirecionado do servidor Django)
\item Base de dados não-relacional Redis.
\item Python, versão 2.7 (Usando as \textit{frameworks flask} e \textit{Django})
\item IDE de Python (\textit{Pycharm Professional}).
\item Contentores para virtualização: Docker \& Kubernetes.
\item Web Interface: HTML 5, Bootstrap, Javascript.
\item Firewall e IDS: IPtables \& Snort
\end{itemize}


\subsubsection{Recursos de hardware:}
\begin{itemize}
\item Google Cloud Cluster - GCE (Single Instances for DB and CDN)
\item Computadores da ADMIN-FCUL para testes e servidores locais
\end{itemize}

%---------------------------------------------------------------------------------------------

\chapter{Conclusão}

Perante o projecto que nos foi proposto, definimos os requisitos funcionais e não funcionais como pilares da nossa proposta de trabalho.
\\

\noindent Através de uma pesquisa aos \textit {websites} referidos na introdução e um conjunto de boas práticas de serviços \textit {web}, adicionamos funcionalidades possíveis de implementar no SI, e que determinam uma plataforma única de aceder aos conteúdos propostos no projecto. \\

\noindent No entanto, achamos relevante referir que esta proposta é um pouco irrealista, tendo em conta o tempo disponivel para a sua realização.

%--------------------------------------------------------------------------------------------- Refazer

\begin{thebibliography}{1}

\bibitem{notes} Leslie Lamport {\em LaTEX: a document preparation system,}
  2nd edition, 1994.

\bibitem{impj} Roger S- Pressman, Bruce Maxim, {\em Software Engineering: A pratitioner's Appoach,} McGraw-Hill, 8ª edição, 1973.

\bibitem{norman} Página web \textit{soundcloud} {\em https://soundcloud.com/, } 2016.

\bibitem{fo} Página web \textit{bandcamp} {\em https://bandcamp.com/,} 2016.

\bibitem{fo} Página web \textit{ticketline} {\em http://ticketline.sapo.pt/en/,} 2016.

\bibitem{fo} Página web \textit{ticketmaster} {\em http://www.ticketmaster.com/,} 2016.

\bibitem{fo} Página web \textit{spotify} {\em https://play.spotify.com/,} 2016.

\bibitem{python} Página web sobre Python {\em https://docs.python.org/2/}, 2016. (Documentação Python 2.7).

\bibitem{java} Página web sobre Java {\em https://docs.oracle.com/javase/8}, 2016 ( Documentação Java SE 8).

\bibitem{amazon} \textit{Google Cloud Computing} {\em https://cloud.google.com/}, 2016.

\bibitem{encript} Let's Encript {\em https://letsencrypt.org/}, 2016. (Certificados SSL)

\end{thebibliography}

\clearpage

\end{document}