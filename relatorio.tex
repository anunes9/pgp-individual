 %formatacao do documento e tipo do documento
\documentclass[12pt, a4paper, twoside]{article}

%pacotes de extensoes
\usepackage[portuges]{babel} %pkg da lingua portugues
\usepackage[latin1,utf8]{inputenc} %pkg da lingua portugues
\usepackage{titlepic} %inserir imagens na capa
\usepackage{graphicx}
\usepackage[table,xcdraw]{xcolor} %usar fomatacao nas tabelas
\usepackage{geometry}\geometry{a4paper=true,portrait=true,left=3cm,right=3cm,top=2.5cm,bottom=3.5cm}
\usepackage{multicol}
\usepackage{hyphenat}
\usepackage{booktabs}

\makeindex

%inicio do documento
\begin{document}

%inserir capa / titulo
\title{
Departamento de Informática\\Licenciatura em Tecnologias de Informação\\
\vskip 1 cm
\includegraphics[height=5 cm]{logo_fcul.png}
\vskip 1 cm
Planeamento e Gestão de Projecto\\
\vskip 1 cm
Relatório Projecto individual}
\author{André Nunes\\43304}
\date{}

\maketitle

\newpage
%indice
\tableofcontents

\pagenumbering{roman}
\setcounter{page}{1}

\newpage
\pagenumbering{arabic}
\setcounter{page}{1}
\section{Introdução}
\vskip 1 cm

Este projeto tem como objetivo proporcionar ao aluno uma primeira experiência de planeamento e gestão de um projeto de software. Nesse sentido, a ênfase da execução do projeto deve ser colocada nas tarefas de planeamento e gestão, e não nas tarefas de desenvolvimento (ou seja, não se espera que o produto de software entregue no final do projeto apresente elevados níveis de eficácia e eficiência).\\

\textbf{Produto}\\

Pretende-se que o aluno desenvolva uma aplicação Web para registo de objetos emprestados pelo utilizador. No mínimo, a aplicação deve realizar as seguintes funcionalidades:
\begin{itemize}
\item registar um empréstimo de um objeto a indicar, numa data a indicar, a uma pessoa a indicar;
\item registar a devolução de um objeto emprestado, numa data a indicar;
\item ver uma lista de objetos emprestados atualmente.
\end{itemize}

O aluno poderá acrescentar mais funcionalidades se assim o entender. A aplicação Web não deve fazer uso de uma base de dados externa.

\newpage
\section{Planeamento das tarefas}
\vskip 1 cm

Este projeto foi dividido em cinco fases de implementação:

\\
\noindent
\begin{enumerate}
	\item Desenvolver Base
  	\begin{itemize}
		\item Servidor Flask
  		\item Página inicial (html + css)
  	\end{itemize}
  	\item Funcionalidade Registar Empréstimo (Reg Emp)
  	\begin{itemize}
  		\item Html + css (a)
  		\item Javascript (b)
  		\item Ligação ao servidor (c)
	\end{itemize}
	\item Funcionalidade Registar Devolução (Reg Dev)
  	\begin{itemize}
  		\item Html + css (a)
  		\item Javascript (b)
  		\item Ligação ao servidor (c)
	\end{itemize}
	\item Funcionalidade Listar Empréstimos (Listar Emp)
  	\begin{itemize}
  		\item Html + css (a)
  		\item Javascript (b)
  		\item Ligação ao servidor (c)
	\end{itemize}
	\item Desenvolver Base
  	\begin{itemize}
  		\item Testes
  		\item Documentação
	\end{itemize}
\end{enumerate}

\newpage
Para cada tarefa foi planeado um tempo de execução, apresentado na tabela seguinte:
\begin{table}[h]
\centering
\begin{tabular}{l|c|c}

Tarefa 			& Tempo Planeado (horas) & Tempo Usado (horas) \\ \hline
Servidor Flask  & 2              & 1           \\
Página incial   & 2              & 1           \\
Reg Emp (a)     & 2              & 1           \\
Reg Emp (b)     & 2              & 1           \\
Reg Emp (c)     & 2              & 1           \\
Reg Dev (a)     & 2              & 1           \\
Reg Dev (b)     & 2              & 1           \\
Reg Dev (c)     & 2              & 1           \\
Listar Emp (a)  & 2              & 1           \\
Listar Emp (b)  & 2              & 1           \\
Listar Emp (c)  & 2              & 1           \\
Testes          & 2              & 1           \\
Documentação    & 1              & 1          

\end{tabular}
\\~\\
\caption{Tabela de tempo de cada tarefa}
\end{table}

cenas


\clearpage
\newpage
\section{Conclusão}

O objetivo principal deste relatório é a construção e a aplicação de um modelo de otimização do número de funcionários da empresa Zuloc de modo a otimizar a sua produção de vacinas e manter lucro.\\
Com a análise da primeira fase, caso em que o número de funcionários é 48, podemos concluir que a empresa não estaria a ser capaz de produzir as vacinas necessárias para o número de encomendas que recebe por isso esse número de funcionários não é o ideal.\\
Com a análise da segunda fase, caso em que o número de funcionários foi otimizado para 193, podemos concluir que com esse número de funcionários a empresa já seria capaz de cumprir com o número de vacinas necessárias para as encomendas recebidas e mesmo assim manter uma taxa de ocupação de mão de obra acima dos 80\%.\\
Concluímos então que com a contratação de 145 novos empregados a empresa Zuloc iria beneficiar de um aumento de capacidade de produção de vacinas e assim concluir todas as encomendas recebidas sem deixar encomendas em atraso.

%fim do documento
\end{document}